\documentclass{article}
\usepackage[utf8]{inputenc}
\usepackage{polski}
\usepackage[polish]{babel}

\begin{document}

\title{POP - projekt}
\author{Adam Czupryński \and Szymon Makuch}
\maketitle

\section{Temat}
Ewolucja różnicowa z modyfikacją wdrażającą (nieszablonowy) model zastępczy (surrogate model) w celu wyboru osobników z populacji do określenia ich jakości za pomocą funkcji celu.

\section{Interpretacja tematu}
Zadanie polega na implementacji modelu zastępczego w algorytmie ewolucji różnicowej, służącego do oszczędnej selekcji osobników do ewaluacji. Głównym celem jest ograniczenie liczby kosztownych obliczeniowo ewaluacji funkcji celu przy zachowaniu efektywności algorytmu.

\section{Propozycja rozwiązania}
W celu ograniczenia kosztów obliczeniowych związanych z ewaluacją wszystkich osobników w populacji użyjemy algorytmu ID3 jako modelu zastępczego, uczonego na podstawie historycznych ewaluacji. ID3 umożliwia tworzenie drzew decyzyjnych, na podstawie których łatwo będzie wybrać najlepszych osobników do ewaluacji.

\section{Implementacja i środowisko}
Projekt zostanie zaimplementowany w języku Python z wykorzystaniem bibliotek:
\begin{itemize}
    \item scipy - implementacja algorytmu ewolucji różnicowej
    \item scikit-learn - testy i metryki
    \item matplotlib, seaborn i pandas - wizualizacja wyników
\end{itemize}

\section{Założenia}
Algorytm ewolucji różnicowej zostanie zaczerpnięty z biblioteki scipy. Wyniki testów zostaną przedstawione na wykresach.

\section{Badanie jakości}
Algorytm testowany będzie na standardowych funkcjach testowych (np. Rastrigin, Rosenbrock), dla każdej z funkcji zastosowana zostanie k-krotna walidacja krzyżowa. Badania jakości obejmować będą testy liczby wywołań funkcji celu, jakości znalezionych rozwiązań czy czasu działania.

\end{document}